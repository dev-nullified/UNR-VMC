\section{Project Description}
\label{sect:project_description}

The Veteran Military Center Tracking and Analytics Program (VMC-TAP for short) is intended to be an easy to use data ingestion and analytical platform for the Veteran Military Center (VMC for short) at the University of Nevada, Reno. Team 20  was approached by the administrators of the VMC to create reporting and analytics. The task of reporting and analytic used to be the responsibility of a single individual hired by the center. Due to budget constraints, the VMC could not afford to keep this individual. This was a problem for the center as they had to use their limited man hours. Attendance and utilization reporting are extremely important for the center as it informs administrators how the center is doing. 

Team 20 created a couple of goals for this project. The first goal is to make the application easy to use via an intuitive and simple interface. Our thinking was that if the interface was clunky and hard to navigate, the VMC and/or other potential clients would be less inclined to use VMC-TAP. The second goal was to provide meaningful reports. An example would be a report showing how many students haven’t visited the VMC in over 30 days. With this information, the VMC could reach out to the student to see how they are doing and/or how the VMC could better meet their needs. 
	
The main function of VMC-TAP is to provide actionable information based on the attendance data that the Veteran Military Center currently collects. Attendance data is collected by the center through an iPad at the front desk and through portable scanners for events that occur outside the center. This data only collects the card number on the Wolf Card. The scan data is then sent off to the office of Data and Analysis who then returns an excel file to the VMC administrators. This excel file  contains the demographic data of scanned students. The VMC administrators would then upload the data from office of Data and Analysis into VMC-TAP via the program’s desktop application. VMC-TAP will then to parse the data, organize it, store it in the database. The program will store the ingested data over time allowing the center administrators to preform trend analysis. VMC administrators will then be able to generate reports based on the stored data. With VMC-TAP, administrators of the Veteran Military Center will have instant access to powerful reports that once took them hours to generate.

There is a lot that can be done to extend the project in the future. A web interface, for instance, can be built to make the project more accessible from mobile devices. At the moment we only plan on making a desktop application. A web interface would also allow VMC admins to have a unified dashboard that’s consistent across all environments (desktop and mobile). The project can also be extended by allowing users to create custom reports. Currently, VMC-TAP will only be able to run a pre-defined list of reports. Functionality could be added to allow VMC administrators to change the criteria of the reports without having to add new functions to the code-base. These features would be a welcomed addition to the project and its sponsors.

One limitation Team 20 faces is that they won’t have access to real data to test against due to FERPA. Team 20 is able to get around this by testing with data collected from each team member’s information. Another limitation is the skill level of the team. Some have dabbled in both Python and SQL separately, but they’ve never created a working program using those two technologies together. The team is brushing up on those skills through online tutorials and textbooks. Team 20 is confident that their limitations are detrimental to the project and instead view it as an opportunity for professional growth.

VMC-TAP will be built in Python 3.7 due to its long-term support cycle and ease of use. The project will also be using SQLite for the database. This was decided because Team 20 didn't want to worry about administering a database management system. For libraries, the project will use matplotlib for data visualization, tk or PyQt5 for UI, openpyxl for data ingestion, and sqlite3 for database interaction. These technologies are just what's required to give VMC-TAP a solid foundation to build upon. It is expected that additional libraries will be added over-time to expand the projects capabilites.

Current advisors for VMC-TAP and Team 20 include Nikkolas Irwin who is mentoring the team and Dr. Harris who is their faculty advisor. Project sponsors include Chai Cook, the Assistant Director of Veteran Services at UNR and John Pratt, a VA School Certifying Official and administrator of the Veteran Militery Center. Devrin Lee is the advising professor in CS 425 for Team 20.

Members of Team 20 include Autumn Cuellar, Joseph Yott, and Brandon Freshour. Autumn Cuellar is a hard working student that works in a Robotics Lab at the University of Nevada, Reno. She has a little experience with Python, and is good at making sure assignments and tasks are done on time. Brandon Freshour is an IT professional with 5 years of experience in the field. His experience ranges from managing servers across the globe in diverse environments, managing databases, and creating web applications for clients. Joseph Yott is a U.S. Army veteran with experience in software and video game development. With skills in team management, Joe motivates and guides his team through the process of making software. 

Members of Team 20 will be able to work in a team in a professional software engineering project by completing this project. In addition, members of Team 20 also gain hard skill of front end and back end development. 


% EOF